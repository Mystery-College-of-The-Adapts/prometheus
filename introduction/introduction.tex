%<a short general introduction>
The computational task of \textit{finding patterns on strings} is one of the most well studied problem in Computer Science. This could be attributed to the high applicability of this very generic problem to various fields of study like biology, chemistry, economics, the social sciences and even music. The continuous trend nowadays of searching patterns through data sets with large size pushes the edge of knowledge in more efficient and scalable hardware architecture and algorithms. The hardware industry in microchip processing is hitting the physical limits of computing components as predicted by the heuristic Moore's Law. In response to this, mathematicians, physicists, computer scientists and engineers are looking into various areas of computing models which could provide a leap in computing power.

One area of interest to researchers in computing is that of \textit{unconventional} computing models. These are models of machines which derive their data processing capabilities from unconventional means. One such model which is of interest in this study is the \textit{quantum computing} model. This computing model abstracts a machine which computes using hardware which exhibits quantum mechanical effects. These physical effects are not observable with the human eyes like how we view things around us as we see them. Special instrumentations are required in order to observe such effects which are usually in the atomic level. 

The interest in quantum computing may be attributed to the ever growing trend in miniaturization of computing components in microchips. Current computing components in chips are now so small that they only have few atoms in them. Quantum physical effects are more highly observable in the atomic level. These effects may affect the state of computing components during computation. One motivation in studying quantum computation then is how to harness these effects to aid in the computing process instead of shielding the computing process from them. An active research area in quantum computing is the design of algorithms not necessarily for computationally hard problems only but also for computationally easier problems but with a large scope of application such as the problem of pattern matching.

In this thesis, the main goal is to design \textit{efficient quantum algorithms for pattern matching} on the \textit{string} (sequence of symbols) data structure. Though the results in this study are all theoretical in nature, we hope to inspire succeeding researchers at the very least to have the desire to simulate these results on a classical computer and from their results be able to design even better quantum algorithms for the problem of interest.

\section*{Overview of the thesis}

The development of the content of this thesis is arranged as follows. Part I provides the reader with a technical background about quantum computing. Part II is a discussion of the quantum algorithms developed in this study both for the exact and approximate string matching. Part III provides a summary and a general conclusion about the contributions in this work.

\textbf{Part I.} In this part we provide the reader the necessary technical background for understanding the content of the succeeding chapters. We discuss the concept of quantum computing as an unconventional computing model and its relation to the classical model reversible computing. We put into historical context the development of quantum computing from the initial results in reversible computing. We briefly discuss about the some fundamental quantum algorithms which sparked interest in the model. We then present the basic concepts of quantum bits and registers, and some quantum mechanical effects used in designing quantum algorithms. We discuss about the vector representation of quantum states and matrix representation of quantum operators. We then discuss about the quantum circuit model and the quantum gates which are relevant in this work. Lastly, we present briefly some earlier quantum algorithms for exact and approximate string matching.

\textbf{Part II.} In this part we present three quantum algorithms we designed for the exact and approximate string matching problem. In Chapter~\ref{chap:grover} we present a quantum algorithm for exact string matching that follows the quantum algorithm design paradigm of Grover's quantum search algorithm. We discuss details of our earlier work on this quantum algorithm \cite{Aborot2013} and then present new additional results which provides a more detailed analysis of the circuit, time and space complexity of the algorithm. In Chapter~\ref{chap:convolution} we present a quantum algorithm for approximate string matching based on the concept of convolution of two sequences. We also give an analysis of the circuit complexity of the operators used in the quantum convolution algorithm to provide a tighter bound on its time complexity. Lastly, in Chapter~\ref{chap:filtering} we present a quantum algorithm for approximate string matching based on a classical filtering method in \cite{Amir2004}. We provide a quantum circuit construction for the operators used in the algorithm and analyze their circuit complexity to bound the algorithm's total time complexity.

\textbf{Part III.} In this part we summarize the contributions made in this study. We also provide a comparison of our results with that of the referenced quantum algorithms for exact and approximate string matching.

\section*{Contributions of the thesis}
The following is a list of the contributions of the author in this study.
\begin{itemize}

\item A published work of the author and colleagues about a quantum algorithm for the exact string matching problem \cite{Aborot2013} is presented in Chapter~\ref{chap:grover}. It is based on the quantum algorithm design paradigm of Grover's quantum search algorithm for unstructured database \cite{Grover1996}. Additional results since the publication of the work are presented. A quantum circuit design is provided for implementing the oracle function $f(\cdot)$ for comparing subsequences of T with P. A design of a symbol comparator module based in \cite{Thapliyal2010} is used as a component circuit. The comparator quantum circuit for the oracle function $f(\cdot)$ for comparing subsequences in T and P is designed as a parallel implementation of the symbol comparator module. A quantum circuit is also designed for the amplitude amplification-base operator. A circuit complexity analysis of the designed quantum circuits provided a tighter bound on the time complexity of the algorithm. The circuit, time and space complexity of each operator used in the quantum algorithm is summarized in tabular format for quick referencing.


\item A quantum algorithm for the approximate string matching problem which uses the concept of convolution of sequences is presented in Chapter~\ref{chap:convolution}. The concept of convolution of two sequences is discussed and its application to string matching is explained. A sub-routine algorithm for preparing arbitrary quantum superposition states \cite{Rosenbaum2009} used in encoding binary indicator sequences for T and P is discussed and illustrated using an example instance. A quantum analogue of the classical convolution theorem is provided and a unitary operator for approximating point-wise multiplication of two vectors is redefined based on its definition in \cite{Curtis2004}. Using a cycle-based decomposition method of a permutation matrix in \cite{Welch2014b,Welch2014c,Welch2015} a circuit complexity analysis of the unitary point-wise multiplication operator is provided. A comparison of the time and space complexity of the convolution-based quantum algorithm with that of its classical counterpart is also provided. The time complexity of each operator used in the algorithm is summarized in tabular format for quick referencing.

\item A quantum algorithm for the approximate string matching problem which uses a classical filtering technique in \cite{Amir2004} is presented in Chapter~\ref{chap:filtering}. The classical filtering concept is discussed and illustrated through an example instance of the problem. A quantum analogue of the concept is then provided and a quantum algorithm which uses this concept is outlined. Each quantum operator in each step of the algorithm is presented with their corresponding quantum circuit design. Based from the quantum circuit of these operators the circuit complexity of the filtering-based quantum algorithm is calculated. The circuit, time and space complexity of each operator and the whole quantum algorithm is summarized in tabular format for quick referencing.

\end{itemize}

%<a list of contributions of the thesis>

\section*{Selected publications of the author}
The following is a list of published works of the author relevant to the topic of this study.
\begin{itemize}
\item Aborot, J. A., Adorna, H. N., de Jesus, B. K. (2013) Solving the Exact Pattern Matching Problem Constrained to Single Occurrence of Pattern P in String S Using Grover?s Quantum Search Algorithm. Theory and Practice of Computation, Vol. 7, Proceedings in Information and Communications Technology, 124-142
\item Aborot, J. A. (2015) Quantum approximate string matching for large alphabets (to appear) Proceedings of Workshop on Computation: Theory and Practice 2015. University of the Philippines Cebu, Philippines
\end{itemize}

