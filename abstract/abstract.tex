%This thesis follows the investigation on the relationship of membrane computing and
%grammar systems done by Csuhaj-Varju et. al.. We study the
%relationship of multigenerative grammar systems by Meduna and Lukas with
%P systems using string-objects and distributed P systems.
%We present a variant of dP Systems that uses computing strategies based on multigenerative
%grammar systems called Controlled Rewriting dP Systems. We will measure its computational and
%communication complexity, as well as its power in comparison to the classes of languages
%of the Chomsky Hierarchy. Additionaly, this thesis also presents an automata system based
%on multigenerative grammar systems called multirecognizing automata systems, and some
%results and insights on the parallelism of dP Automata.

%In this research, we follow the investigation on the relationship of membrane computing and grammar
%systems initiated by Csuhaj-Varju et. al.. In particular, we study Multigenerative Grammar
%Systems and Distributed P Systems. 

ANTINERO ABOROT, JEFFREY. \uppercase{Quantum algorithms for string matching}. (Under the direction of HENRY N. ADORNA, PhD)
In this thesis we present three quantum algorithms for the exact and approximate string matching problem. Our first algorithm is an amplitude amplification-based quantum algorithm for the exact string matching problem which has time complexity in $\Om\left( \left\lfloor \frac{\pi}{4}\sqrt{N} \right\rfloor \left( \log_2 N + \log_2 \vert \Sigma \vert \right) \right)$ with additional logarithmic terms for error of approximation and space complexity in $\Om\left( NM\log_2\vert \Sigma \vert + M\log_2 \vert \Sigma \vert + M + \log_2 N \right)$. It outputs a solution index in a text with probability $\approx 1$. We provided a quantum circuit construction for the oracle function for comparing sequence of symbols in an alphabet which is absent in the quantum algorithms reviewed for this study. We also provided deeper complexity analysis as compared to reviewed related quantum algorithm for the same problem. 

Our second algorithm is a quantum Fourier transform-based quantum algorithm for the approximate string matching problem with time complexity in $\Om\left( K\left(\log_2^2 (N+M) + q\log_2 (N+M) \right) \right)$ and space complexity in $\Om\left(\vert \Sigma \vert\log_2 (N+M) + KM\log_2 \vert \Sigma \vert \right)$. It returns a solution index in a text with probability proportional to the number of matching symbols between the solution subsequence of the text and pattern.  We used the concept of convolution in digital signal processing to compute the number of matching symbols between two sequences. Lastly, our third algorithm is a quantum algorithm for approximate string matching based on a classical filtering method. It has time complexity in $\Om\left( \log_2 N + \vert \Sigma \vert\log_2 M \right)$ and space complexity in $\Omega\left( N\log_2\vert \Sigma \vert + \log_2 N + \log_2 \vert \Sigma \vert \right)$. The algorithm outputs a solution index with probability proportional to the number of matching symbols between the solution subsequence of the text and the pattern. We designed quantum symbol operators for identifying the first occurrence index of distinct symbols in the pattern. This provides reusability when searching for the same pattern on different input texts which is comparable to the level of reusability provided by a reviewed related quantum algorithm for the same problem. We also provided quantum circuit design for the quantum operators used in each of our algorithms' sub-routines. In comparison to the reviewed related quantum algorithms for the same problem, our algorithms provide lower time complexity. 



 



~\\
\noindent {\bf Keywords:} \textit{Unconventional computing, Quantum computing, String matching, Quantum algorithms, Pattern matching, Computing model}
